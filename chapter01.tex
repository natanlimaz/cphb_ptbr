\chapter{Introdução}

Programação competitiva combina dois tópicos:
(1) o design de algoritmos e (2) a implementação de algoritmos.

O \key{design de algoritmos} consiste em solução de problemas
e pensamento matemático.
São necessárias habilidades para analisar problemas e resolvê-los de forma criativa.
Um algoritmo para resolver um problema deve ser tanto correto quanto eficiente, e o cerne do problema muitas vezes é inventar um algoritmo eficiente.

O conhecimento teórico de algoritmos é importante para programadores competitivos. Tipicamente, uma solução para um problema é uma combinação de técnicas bem conhecidas e novas ideias. As técnicas que aparecem na programação competitiva também formam a base para a pesquisa científica de algoritmos.

A \key{implementação de algoritmos} requer boas habilidades de programação. Na programação competitiva, as soluções são avaliadas testando um algoritmo implementado usando um conjunto de casos de teste. Portanto, não é suficiente que a ideia do algoritmo seja correta, mas a implementação também deve ser correta.

Um bom estilo de codificação em competições é direto e conciso. Os programas devem ser escritos rapidamente, porque não há muito tempo disponível. Ao contrário da engenharia de \emph{software} tradicional, os programas são curtos (geralmente com no máximo algumas centenas de linhas de código) e dispensam manutenção após a competição.

\section{Linguagens de programação}

\index{linguagem de programação}

Atualmente, as linguagens de programação mais populares usadas em competições são C++, Python e Java.
 Por exemplo, no Google Code Jam 2017,
 entre os 3.000 melhores participantes,
79\% usaram C++,
16\% usaram Python and
8\% usaram Java \cite{goo17}.
Alguns participantes também usaram outras linguagens.

Muitas pessoas pensam que C++ é a melhor escolha para um programador competitivo, e o C++ está quase sempre disponível nos sistemas de competição. Os benefícios de usar C++ são ser uma linguagem muito eficiente e contar com uma biblioteca padrão com uma grande coleção de estruturas de dados e algoritmos.

Por outro lado, é bom dominar várias linguagens e entender suas forças. Por exemplo, se inteiros grandes são necessários para um problema, Python pode ser uma boa escolha, porque contém operações embutidas para cálculos com inteiros grandes. Ainda assim, a maioria dos problemas em competições de programação são definidos de forma que o uso de uma linguagem de programação específica não crie uma vantagem injusta.

Todos os exemplos de programas neste livro são escritos em C++ e as estruturas de dados e algoritmos da biblioteca padrão são frequentemente usados. Os programas seguem o padrão C++11, que pode ser usado na maioria das competições hoje em dia. Se você ainda não consegue programar em C++, agora é um bom momento para começar a aprender.

\subsubsection{Esboço de código em C++}

Um esboço de código típico em C++ para programação competitiva se parece com isso:

\begin{lstlisting}
#include <bits/stdc++.h>

using namespace std;

int main() {
    // solucao vai aqui
}
\end{lstlisting}

A linha \texttt{\#include} no inicio do código é uma funcionalidade do compilador \texttt{g++} que nos permite incluir toda a biblioteca padrão. Assim, não é necessário incluir separadamente bibliotecas como \texttt{iostream}, \texttt{vector} e \texttt{algorithm}, mas elas ficam disponíveis automaticamente.

A linha \texttt{using} declara que as classes e funções da biblioteca padrão podem ser usadas diretamente no código. Sem a linha \texttt{using}, teríamos que escrever, por exemplo, \texttt{std::cout}, mas agora basta escrever \texttt{cout}.

O código pode ser compilado usando o seguinte comando:

\begin{lstlisting}
g++ -std=c++11 -O2 -Wall test.cpp -o test
\end{lstlisting}

Este comando produz um arquivo binário \texttt{test} a partir do código-fonte \texttt{test.cpp}. O compilador segue o padrão C++11 (\texttt{-std=c++11}), otimiza o código (\texttt{-O2}) e exibe avisos sobre possíveis erros (\texttt{-Wall}).

\section{Entrada e saída}

\index{entrada e saída}

Na maioria das competições, comandos padrões são usados para ler a entrada e escrever a saída. Em C++, os comandos padrões são \texttt{cin} para entrada e \texttt{cout} para saída. Além disso, as funções em C \texttt{scanf} e \texttt{printf} podem ser usadas.

A entrada para o programa geralmente consiste de números e strings que são separados por espaços e novas linhas. Eles podem ser lidos pelo comando \texttt{cin} da seguinte forma:

\begin{lstlisting}
int a, b;
string x;
cin >> a >> b >> x;
\end{lstlisting}

Esse tipo de código sempre funciona,
assumindo que há pelo menos um espaço
ou uma quebra de linha entre cada elemento da entrada.
Por exemplo, o código acima pode ler
ambas as entradas a seguir:
\begin{lstlisting}
123 456 monkey
\end{lstlisting}
\begin{lstlisting}
123    456
monkey
\end{lstlisting}
O comando \texttt{cout} é usado para saída
da seguinte forma:
\begin{lstlisting}
int a = 123, b = 456;
string x = "monkey";
cout << a << " " << b << " " << x << "\n";
\end{lstlisting}

As entradas e saídas às vezes são um gargalo no programa. As seguintes linhas no início do código tornam as entradas e saídas mais eficientes.

\begin{lstlisting}
ios::sync_with_stdio(0);
cin.tie(0);
\end{lstlisting}

Note que a quebra de linha \texttt{"\textbackslash n"} é mais rápida do que o \texttt{endl}, porque o \texttt{endl} sempre força uma operação de \emph{flush}.

As funções \texttt{scanf} e \texttt{printf} da linguagem C, são uma alternativa aos comandos padrões do C++. Elas são geralmente um pouco mais rápidas, mas também são mais difíceis de usar. O código seguinte lê dois números inteiros da entrada:
\begin{lstlisting}
int a, b;
scanf("%d %d", &a, &b);
\end{lstlisting}
O código seguinte imprime dois números inteiros:
\begin{lstlisting}
int a = 123, b = 456;
printf("%d %d\n", a, b);
\end{lstlisting}

Às vezes, o programa deve ler uma linha inteira da entrada, possivelmente contendo espaços. Isso pode ser feito usando a função \texttt{getline}:

\begin{lstlisting}
string s;
getline(cin, s);
\end{lstlisting}

Se a quantidade de dados for desconhecida, o seguinte laço é útil:
\begin{lstlisting}
while (cin >> x) {
    // codigo
}
\end{lstlisting}
Este laço lê elementos da entrada um após o outro, até que não haja mais dados disponíveis na entrada.

Em alguns sistemas de competições, arquivos são usados para entrada e saída. Uma solução simples para isso é escrever o código como de costume usando comandos padrões, mas adicionar as seguintes linhas no início do código:
\begin{lstlisting}
freopen("input.txt", "r", stdin);
freopen("output.txt", "w", stdout);
\end{lstlisting}
Depois disso, o programa lê a entrada do arquivo ''input.txt'' e escreve a saída para o arquivo ''output.txt''.

\section{Trabalhando com números}

\index{inteiro}

\subsubsection{Inteiros}

O tipo inteiro mais utilizado em programação competitiva é o \texttt{int}, que é um tipo de 32 bits com uma faixa de valores de $-2^{31} \ldots 2^{31}-1$, ou cerca de $-2 \cdot 10^9 \ldots 2 \cdot 10^9$. Se o tipo \texttt{int} não for suficiente, o tipo de 64 bits \texttt{long long} pode ser utilizado. Ele possui uma faixa de valores de $-2^{63} \ldots 2^{63}-1$, ou aproximadamente $-9 \cdot 10^{18} \ldots 9 \cdot 10^{18}$.

O código a seguir define uma variável do tipo \texttt{long long}:
\begin{lstlisting}
long long x = 123456789123456789LL;
\end{lstlisting}
O sufixo \texttt{LL} significa que o tipo do número é \texttt{long long}.

Um erro comum ao usar o tipo \texttt{long long}
é que o tipo \texttt{int} ainda é usado em algum lugar do código.
Por exemplo, o seguinte código contém um erro sutil:

\begin{lstlisting}
int a = 123456789;
long long b = a*a;
cout << b << "\n"; // -1757895751
\end{lstlisting}

Embora a variável \texttt{b} seja do tipo \texttt{long long},
ambos os números na expressão \texttt{a*a} são do tipo \texttt{int} e o resultado também é do tipo \texttt{int}.
Devido a isso, a variável \texttt{b} conterá um resultado incorreto.
O problema pode ser resolvido alterando o tipo de \texttt{a} para \texttt{long long} ou alterando a expressão para \texttt{(long long)a*a}.

Normalmente, os problemas de competição são definidos de forma que o tipo \texttt{long long} seja suficiente. Ainda assim, é bom saber que o compilador \texttt{g++} também oferece um tipo de 128 bits chamado \texttt{\_\_int128\_t} com uma faixa de valores de $-2^{127} \ldots 2^{127}-1$, ou aproximadamente $-10^{38} \ldots 10^{38}$. No entanto, este tipo não está disponível em todos os sistemas de competição.

\subsubsection{Aritmética modular}

\index{resto}
\index{aritmética modular}

Denotamos por $x \bmod m$ o resto da divisão de $x$ por $m$. Por exemplo, $17 \bmod 5 = 2$, porque $17 = 3 \cdot 5 + 2$.

Às vezes, a resposta para um problema é um número muito grande, mas é suficiente para imprimir o "módulo $m$", ou seja, o resto quando a resposta é dividida por $m$ (por exemplo, "módulo $10^9+7$"). A ideia é que, mesmo que a resposta real seja muito grande, é suficiente usar os tipos \texttt{int} e \texttt{long long}.

Uma propriedade importante do resto é que,
na adição, subtração e multiplicação,
o resto pode ser obtido antes da operação:

\[
\begin{array}{rcr}
(a+b) \bmod m & = & (a \bmod m + b \bmod m) \bmod m \\
(a-b) \bmod m & = & (a \bmod m - b \bmod m) \bmod m \\
(a \cdot b) \bmod m & = & (a \bmod m \cdot b \bmod m) \bmod m
\end{array}
\]

Assim, podemos obter o resto após cada operação e os números nunca se tornarão muito grandes.

Por exemplo, o código seguinte calcula $n!$, 
o fatorial de $n$, módulo $m$:
\begin{lstlisting}
long long x = 1;
for (int i = 2; i <= n; i++) {
    x = (x*i)%m;
}
cout << x%m << "\n";
\end{lstlisting}

Normalmente, queremos que o resto esteja sempre entre $0\ldots m-1$.
No entanto, em C++ e em outras linguagens, o resto de um número negativo é zero ou negativo.
Uma maneira fácil de garantir que não haja restos negativos é primeiro calcular o resto como de costume e depois adicionar $m$ se o resultado for negativo:
\begin{lstlisting}
x = x%m;
if (x < 0) x += m;
\end{lstlisting}
No entanto, isso só é necessário quando há subtrações no código e o resto pode se tornar negativo.

\subsubsection{Números com ponto flutuante}

\index{números com ponto flutuante}

Os tipos usuais de números de ponto flutuante em programação competitiva são o \texttt{double} de 64 bits e, como uma extensão no compilador \texttt{g++}, o \texttt{long double} de 80 bits. Na maioria dos casos, o tipo \texttt{double} é suficiente, mas o \texttt{long double} é mais preciso.

A precisão necessária da resposta geralmente é fornecida no enunciado do problema. Uma maneira fácil de imprimir a resposta é usar a função \texttt{printf} e fornecer o número de casas decimais na string de formatação. Por exemplo, o código seguinte imprime o valor de $x$ com 9 casas decimais:

\begin{lstlisting}
printf("%.9f\n", x);
\end{lstlisting}

Uma dificuldade ao usar números de ponto flutuante é que alguns números não podem ser representados com precisão como números de ponto flutuante, o que resultará em erros de arredondamento. Por exemplo, o resultado do código seguinte é surpreendente:

\begin{lstlisting}
double x = 0.3*3+0.1;
printf("%.20f\n", x); // 0.99999999999999988898
\end{lstlisting}

Devido a um erro de arredondamento, o valor de \texttt{x} é um pouco menor do que 1, enquanto o valor correto seria 1.

É arriscado comparar números de ponto flutuante com o operador \texttt{==}, pois é possível que os valores devam ser iguais, mas não são devido a erros de precisão. Uma maneira melhor de comparar números de ponto flutuante é assumir que dois números são iguais se a diferença entre eles for menor que $\varepsilon$, onde $\varepsilon$ é um número pequeno.

Na prática, os números podem ser comparados da seguinte forma ($\varepsilon=10^{-9}$):

\begin{lstlisting}
if (abs(a-b) < 1e-9) {
    // a e b sao iguais
}
\end{lstlisting}

Observe que, embora os números de ponto flutuante sejam imprecisos, inteiros até um certo limite ainda podem ser representados com precisão. Por exemplo, usando \texttt{double}, é possível representar com precisão todos os inteiros cujo valor absoluto é no máximo $2^{53}$.

\section{Encurtando código}

Códigos curtos são ideais em programação competitiva,
porque os programas devem ser escritos o mais rápido possível.
Por causa disso, os programadores competitivos geralmente definem
nomes mais curtos para tipos de dados e outras partes do código.

\subsubsection{Nomes para tipos}
\index{tuppdef@\texttt{typedef}}
Usando o comando \texttt{typedef}
é possível dar um nome mais curto para um tipo de dados.
Por exemplo, o nome \texttt{long long} é longo,
então podemos definir o nome mais curto \texttt{ll}:
\begin{lstlisting}
typedef long long ll;
\end{lstlisting}
Depois disso, o código
\begin{lstlisting}
long long a = 123456789;
long long b = 987654321;
cout << a*b << "\n";
\end{lstlisting}
pode ser encurtado como segue:
\begin{lstlisting}
ll a = 123456789;
ll b = 987654321;
cout << a*b << "\n";
\end{lstlisting}

O comando \texttt{typedef}
pode também ser usado com tipos de dados mais complexos.
Por exemplo, o código seguinte dá
o nome \texttt{vi} para um vetor de inteiros
e o nome \texttt{pi} para um pair
que contém dois inteiros.
\begin{lstlisting}
typedef vector<int> vi;
typedef pair<int,int> pi;
\end{lstlisting}

\subsubsection{Macros}
\index{macro}
Outro jeito de encurtar o código é definindo \key{macros}.
Um macro significa que certas strings no código serão mudadas antes da compilação.
Em C++, macros são definidos usando a palavra-chave \texttt{\#define}.

Por exemplo, podemos definir os seguintes macros:
\begin{lstlisting}
#define F first
#define S second
#define PB push_back
#define MP make_pair
\end{lstlisting}
Depois disso, o código
\begin{lstlisting}
v.push_back(make_pair(y1,x1));
v.push_back(make_pair(y2,x2));
int d = v[i].first+v[i].second;
\end{lstlisting}
pode ser encurtado como segue:
\begin{lstlisting}
v.PB(MP(y1,x1));
v.PB(MP(y2,x2));
int d = v[i].F+v[i].S;
\end{lstlisting}

Um macro pode também ter parâmetros que possibilitam encurtar laços e outras estruturas.
Por exemplo, podemos definir o seguinte macro:
\begin{lstlisting}
#define REP(i,a,b) for (int i = a; i <= b; i++)
\end{lstlisting}
Depois disso, o código
\begin{lstlisting}
for (int i = 1; i <= n; i++) {
    search(i);
}
\end{lstlisting}
pode ser encurtado como segue:
\begin{lstlisting}
REP(i,1,n) {
    search(i);
}
\end{lstlisting}

De vez em quando, macros causam bugs que podem ser difíceis de detectar.
Por exemplo, considere o seguinte macro que calcula o quadrado de um número:
\begin{lstlisting}
#define SQ(a) a*a
\end{lstlisting}
O macro \emph{nem sempre} funciona como esperado.
Por exemplo, o código
\begin{lstlisting}
cout << SQ(3+3) << "\n";
\end{lstlisting}
corresponde ao código
\begin{lstlisting}
cout << 3+3*3+3 << "\n"; // 15
\end{lstlisting}

Uma versão melhor do macro é como segue:
\begin{lstlisting}
#define SQ(a) (a)*(a)
\end{lstlisting}
Agora o código
\begin{lstlisting}
cout << SQ(3+3) << "\n";
\end{lstlisting}
corresponde ao código
\begin{lstlisting}
cout << (3+3)*(3+3) << "\n"; // 36
\end{lstlisting}


\section{Matemática}

A matemática desempenha um papel importante nas competições de programação,
e não é possível se tornar um programador competitivo de sucesso sem
ter boas habilidades matemáticas.
Essa seção discute alguns conceitos matemáticos importantes e fórmulas que
serão necessárias mais adiante no livro.

\subsubsection{Fórmulas de soma}

Cada soma da forma
\[\sum_{x=1}^n x^k = 1^k+2^k+3^k+\ldots+n^k,\]
onde $k$ é um inteiro positivo,
tem uma fórmula de forma fechada que é um polinômio de grau $k+1$.
Por exemplo\footnote{\index{fórmula de Faulhaber}
Existe uma fórmula mais geral para somas, chamada de \key{fórmula de Faulhaber},
mas ela é muito complexa para ser apresentada aqui.},
\[\sum_{x=1}^n x = 1+2+3+\ldots+n = \frac{n(n+1)}{2}\]
e
\[\sum_{x=1}^n x^2 = 1^2+2^2+3^2+\ldots+n^2 = \frac{n(n+1)(2n+1)}{6}.\]

Uma \key{progressão aritmética} é uma \index{progressão aritmética}
sequência de números onde a diferença entre quaisquer dois números consecutivos
é constante.
Por exemplo,
\[3, 7, 11, 15\]
é uma progressão aritmética com constante 4.
A soma de uma progressão aritmética pode ser calculada
usando a fórmula
\[\underbrace{a + \cdots + b}_{n \,\, \textrm{números}} = \frac{n(a+b)}{2}\]
onde $a$ é o primeiro número,
$b$ é o último número e
$n$ é a quantidade de números.
Por exemplo,
\[3+7+11+15=\frac{4 \cdot (3+15)}{2} = 36.\]
A fórmula é baseada no fato que a soma consiste de $n$ números e o valor de cada
número é $(a+b)/2$ em média.

\index{progressão geométrica}
A \key{progressão geométrica} é uma sequência de números onde a razão entre
quaisquer dois números consecutivos é constante.
Por exemplo,
\[3,6,12,24\]
é uma progressão geométrica com constante 2.
A soma de uma progressão geométrica pode ser calculada usando a fórmula
\[a + ak + ak^2 + \cdots + b = \frac{bk-a}{k-1}\]
onde $a$ é o primeiro número,
$b$ é o último número e a razão entre números consecutivos é $k$.
Por exemplo,
\[3+6+12+24=\frac{24 \cdot 2 - 3}{2-1} = 45.\]

Esta fórmula pode ser derivada como segue. Seja
\[ S = a + ak + ak^2 + \cdots + b .\]
Multiplicando ambos os lados por $k$, obtemos
\[ kS = ak + ak^2 + ak^3 + \cdots + bk,\]
e resolvendo a equação
\[ kS-S = bk-a\]
obtemos a fórmula.

Um caso especial da soma de progressão geométrica é a fórmula
\[1+2+4+8+\ldots+2^{n-1}=2^n-1.\]

\index{soma harmônica}

A \key{soma harmônica} é uma soma da forma
\[ \sum_{x=1}^n \frac{1}{x} = 1+\frac{1}{2}+\frac{1}{3}+\ldots+\frac{1}{n}.\]

Um limite superior para uma soma harmônica é $\log_2(n)+1$.
Ou seja, podemos modificar cada termo $1/k$ para que $k$ se torna
a potência de dois mais próxima que não excede $k$.
Por exemplo, quando $n=6$, pomode estimar a soma como segue:
\[ 1+\frac{1}{2}+\frac{1}{3}+\frac{1}{4}+\frac{1}{5}+\frac{1}{6} \le
1+\frac{1}{2}+\frac{1}{2}+\frac{1}{4}+\frac{1}{4}+\frac{1}{4}.\]
Este limite superior consiste de $\log_2(n)+1$ partes
($1$, $2 \cdot 1/2$, $4 \cdot 1/4$, etc.),
e o valor de cada parte é no máximo 1.

\subsubsection{Teoria dos conjuntos}

\index{teoria dos conjuntos}
\index{conjunto}
\index{intersecção}
\index{união}
\index{diferença}
\index{subconjuntos}
\index{conjunto universo}
\index{complemento}

Um \key{conjunto} é uma coleção de elementos.
Por exemplo, o conjunto
\[X=\{2,4,7\}\]
contém os elementos 2, 4 e 7.
O símbolo $\emptyset$ denota um conjunto vazio,
e $|S|$ denota o tamanho do conjunto $S$,
ou seja, o número de elementos no conjunto.
Por exemplo, no conjunto acima, $|X|=3$.

Se um conjunto $S$ contém um elemento $x$,
nós escrevemos que $x \in S$,
e senão escrevemos que $x \notin S$.
Por exemplo, no conjunto acima
\[4 \in X \hspace{10px}\textrm{e}\hspace{10px} 5 \notin X.\]

\begin{samepage}
Agora conjuntos podem ser construídos usando operações de conjuntos:
\begin{itemize}
\item A \key{intersecção} $A \cap B$ consiste dos elementos que estão em ambos
$A$ e $B$.
Por exemplo, se $A=\{1,2,5\}$ e $B=\{2,4\}$,
então $A \cap B = \{2\}$.
\item A \key{união} $A \cup B$ consiste dos elementos
que estão em $A$ ou $B$ ou em ambos.
Por exemplo, se $A=\{3,7\}$ e $B=\{2,3,8\}$,
então $A \cup B = \{2,3,7,8\}$.
\item O \key{complemento} $\bar A$ consiste dos elementos
que não estão em $A$.
A interpretação de um complemento depende do
\key{conjunto universo}, que contém todos os elementos possíveis.
Por exemplo, se $A=\{1,2,5,7\}$ e o conjunto universo é
$\{1,2,\ldots,10\}$, então $\bar A = \{3,4,6,8,9,10\}$.
\item A \key{diferença} $A \setminus B = A \cap \bar B$
consiste dos elementos que estão em $A$ mas não estão em $B$.
Note que $B$ pode conter elementos que não estão em $A$.
Por exemplo, se $A=\{2,3,7,8\}$ e $B=\{3,5,8\}$,
então $A \setminus B = \{2,7\}$.
\end{itemize}
\end{samepage}

Se cada elemento de $A$ também pertence a $S$,
dizemos que $A$ é um \key{subconjunto} de $S$,
denotado por $A \subset S$.
Um conjunto $S$ sempre tem $2^{|S|}$ subconjuntos,
incluindo o conjunto vazio.
Por exemplo, os subconjuntos do conjunto $\{2,4,7\}$ são
\begin{center}
$\emptyset$,
$\{2\}$, $\{4\}$, $\{7\}$, $\{2,4\}$, $\{2,7\}$, $\{4,7\}$ e $\{2,4,7\}$.
\end{center}

Alguns conjuntos usados frequentemente são
$\mathbb{N}$ (números naturais),
$\mathbb{Z}$ (inteiros),
$\mathbb{Q}$ (números racionais) e
$\mathbb{R}$ (números reais).
O conjunto $\mathbb{N}$
pode ser definidos de duas maneiras, dependendo
da situação:
como $\mathbb{N}=\{0,1,2,\ldots\}$
ou $\mathbb{N}=\{1,2,3,...\}$.

Podemos também criar um conjunto usando uma regra da forma
\[\{f(n) : n \in S\},\]
onde $f(n)$ é alguma função.
O conjunto contém todos os elementos da forma $f(n)$,
onde $n$ é um elemento em $S$.
Por exemplo, o conjunto
\[X=\{2n : n \in \mathbb{Z}\}\]
contém todos os inteiros pares.

\subsubsection{Lógica}

\index{lógica}
\index{negação}
\index{conjunção}
\index{disjunção}
\index{implicação}
\index{equivalência}

O valor de uma expressão lógica é ou
\key{verdadeiro} (1) ou \key{falso} (0).
Os operadores lógicos mais importantes são
$\lnot$ (\key{negação}),
$\land$ (\key{conjunção}),
$\lor$ (\key{disjunção}),
$\Rightarrow$ (\key{implicação}) e
$\Leftrightarrow$ (\key{equivalência}).
A seguinte tabela mostra o significado destes operadores:

\begin{center}
\begin{tabular}{rr|rrrrrrr}
$A$ & $B$ & $\lnot A$ & $\lnot B$ & $A \land B$ & $A \lor B$ & $A \Rightarrow B$ & $A \Leftrightarrow B$ \\
\hline
0 & 0 & 1 & 1 & 0 & 0 & 1 & 1 \\
0 & 1 & 1 & 0 & 0 & 1 & 1 & 0 \\
1 & 0 & 0 & 1 & 0 & 1 & 0 & 0 \\
1 & 1 & 0 & 0 & 1 & 1 & 1 & 1 \\
\end{tabular}
\end{center}

A expressão $\lnot A$ tém valor oposto do valor de $A$.
A expressão $A \land B$ é verdadeira se ambos $A$ e $B$
são verdadeiros,
e a expressão $A \lor B$ é verdadeira se $A$ ou $B$ ou ambos
são verdadeiros.
A expressão $A \Rightarrow B$ é verdade se quando $A$ for verdadeiro,
$B$ também for verdadeiro.
A expressão $A \Leftrightarrow B$ é verdadeira se $A$ e $B$ ambos forem verdadeiros ou ambos falsos.

\index{predicado}

Um \key{predicado} é uma expressão que é verdadeira ou falsa
dependendo de seus parâmetros.
Predicados são geralmente denotados por letras maiúsculas.
Por exmeplo, podemos definir um predicado $P(x)$
que é verdadeira exatamente quando $x$ é um número primo.
Usando esta definição, $P(7)$ é verdadeiro mas $P(8)$ é falso.

\index{quantificador}

Um \key{quantificador} conecta uma expressão lógica
a elementos de um conjunto.
Os quantificadores mais importantes são
$\forall$ (\key{para todos}) e $\exists$ (\key{existe}).
Por exemplo,
\[\forall x (\exists y (y < x))\]
significa que para cada elemento $x$ no conjunto,
existe um elemento $y$ no conjunto
de tal forma que $y$ é menor que $x$.
Isso é verdadeiro no conjunto dos inteiros,
mas falso no conjunto dos números naturais.

Usando a notação descrita acima,
podemos expressar muitos tipos de proposições lógicas.
Por exemplo,
\[\forall x ((x>1 \land \lnot P(x)) \Rightarrow (\exists a (\exists b (a > 1 \land b > 1 \land x = ab))))\]
significa que se um número $x$ é maior que 1 e não é um número primo,
então existem números $a$ e $b$
que são maiores que $1$ e cujo produto é $x$.
Esta proposição é verdadeira no conjunto dos inteiros.

\subsubsection{Funções}

A função $\lfloor x \rfloor$ arrendonda o número $x$ para baixo, e a função
$\lceil x \rceil$ arrendonda o número $x$ para cima. Por exemplo,
\[ \lfloor 3/2 \rfloor = 1 \hspace{10px} \textrm{e} \hspace{10px} \lceil 3/2 \rceil = 2.\]

As funções $\min(x_1,x_2,\ldots,x_n)$
e $\max(x_1,x_2,\ldots,x_n)$
retornam os menores e maiores valores
$x_1,x_2,\ldots,x_n$.
Por exemplo,
\[ \min(1,2,3)=1 \hspace{10px} \textrm{e} \hspace{10px} \max(1,2,3)=3.\]

\index{fatorial}

O \key{fatorial} $n!$ pode ser definido como
\[\prod_{x=1}^n x = 1 \cdot 2 \cdot 3 \cdot \ldots \cdot n\]
ou recursivamente
\[
\begin{array}{lcl}
0! & = & 1 \\
n! & = & n \cdot (n-1)! \\
\end{array}
\]

\index{número de Fibonacci}

Os \key{números de Fibonacci}
%\footnote{Fibonacci (c. 1175--1250) was an Italian mathematician.}
aparecem em várias situações.
Eles podem ser definidos recursivamente como segue:
\[
\begin{array}{lcl}
f(0) & = & 0 \\
f(1) & = & 1 \\
f(n) & = & f(n-1)+f(n-2) \\
\end{array}
\]
Os primeiros números de Fibonacci são
\[0, 1, 1, 2, 3, 5, 8, 13, 21, 34, 55, \ldots\]
Existe também uma fórmula de forma fechada
para calcular os números de Fibonacci, que é algumas vezes chamada de
\index{fórmula de Binet} \key{fórmula de Binet}:
\[f(n)=\frac{(1 + \sqrt{5})^n - (1-\sqrt{5})^n}{2^n \sqrt{5}}.\]

\subsubsection{Logaritmos}

\index{logaritmo}

O \key{logaritmo} de um número $x$
é denotado $\log_k(x)$, onde $k$ é a base
do logaritmo.
De acordo com esta definição,
$\log_k(x)=a$ exatamente quando $k^a=x$.

Uma propriedade útil dos logaritmos é
que $\log_k(x)$ é equivalente ao número de vezes
necessário para dividir $x$ por $k$ para
alcançar o número 1.
Por exemplo, $\log_2(32)=5$
porque 5 divisões por 2 são necessárias:

\[32 \rightarrow 16 \rightarrow 8 \rightarrow 4 \rightarrow 2 \rightarrow 1 \]

Logaritmos são frequentemente usadas na análise de algoritmos, porque muitos
algoritmos eficientes dividem alguma coisa em cada passo.
Então, podemos estimar a eficiência destes algoritmos usando logaritmos.

O logaritmo de um produto é
\[\log_k(ab) = \log_k(a)+\log_k(b),\]
e consequentemente,
\[\log_k(x^n) = n \cdot \log_k(x).\]
Além disso, o logaritmo de um quociente é
\[\log_k\Big(\frac{a}{b}\Big) = \log_k(a)-\log_k(b).\]
Outra fórmula útil é
\[\log_u(x) = \frac{\log_k(x)}{\log_k(u)},\]
e usando isso, é possível calcular logaritmos para qualquer base se existe
uma maneira de calcular logaritmos para uma base fixa.

\index{logaritmo natural}

O \key{logaritmo natural} $\ln(x)$ de um número $x$
é um logaritmo cuja base é $e \approx 2.71828$.
Outra propriedade de logaritmos é que
o número de dígitos de um inteiro $x$ na base $b$ é
$\lfloor \log_b(x)+1 \rfloor$.
Por exemplo, a representação de
$123$ na base $2$ é 1111011 e
$\lfloor \log_2(123)+1 \rfloor = 7$.

\section{Competições e recursos}

\subsubsection{IOI}

A Olimpíada Internacional de Informática (IOI)
é um concurso anual de programação para
alunos do ensino médio.
Cada país pode enviar uma equipe de
quatro alunos para o concurso.
Geralmente há cerca de 300 participantes
de 80 países.

O IOI consiste em dois concursos de cinco horas de duração.
Em ambos os concursos, os participantes são convidados a
resolver três tarefas algorítmicas de várias dificuldades.
As tarefas são divididas em subtarefas,
cada uma das quais tem uma pontuação atribuída.
Mesmo que os competidores sejam divididos em equipes,
eles competem como indivíduos.

O programa da IOI \cite{iois} regula os tópicos
que podem aparecer em tarefas da IOI.
Quase todos os tópicos do programa IOI são cobertos por este livro.

Os participantes do IOI são selecionados por meio de
concursos nacionais.
Antes do IOI, muitos concursos regionais são organizados,
como a Olimpíada Brasileira de Informática (OBI),
a Olimpíada Báltica de Informática (BOI),
a Olimpíada da Europa Central em Informática (CEOI)
e a Olimpíada de Informática da Ásia-Pacífico (APIO).

Alguns países organizam concursos de prática online
para futuros participantes do IOI,
como o Concurso Aberto da Croácia em Informática \cite{coci}
e a Olimpíada de Computação dos EUA \cite{usaco}.
Além disso, uma grande coleção de problemas de concursos poloneses
está disponível online \cite{main}.

\subsubsection{ICPC}

O Concurso Internacional de Programação Colegiada (ICPC)
é um concurso anual de programação para estudantes universitários.
Cada equipe do concurso é composta por três alunos,
e ao contrário do IOI, os alunos trabalham juntos;
há apenas um computador disponível para cada equipe.

O ICPC é composto por várias etapas, e finalmente o
melhores equipes são convidadas para as Finais Mundiais.
Embora existam dezenas de milhares de participantes
no concurso, há apenas um pequeno número\footnote{O número exato de vagas para as finais
variam de ano para ano; em 2017, havia 133 vagas para a final.} de vagas para as finais disponíveis,
assim, avançar para as finais é uma grande conquista em algumas regiões.

Em cada prova do ICPC, as equipes têm cinco horas para
resolver cerca de dez problemas de algoritmos.
Uma solução para um problema só é aceita se resolver
todos os casos de teste de forma eficiente.
Durante a competição, os competidores poderão visualizar os resultados de outras equipes,
mas na última hora o placar fica congelado e
não é possível ver os resultados das últimas submissões.

Os temas que podem aparecer no ICPC não são tão bem especificados como aqueles no IOI.
De qualquer forma, é claro que mais conhecimento é necessário
no ICPC, especialmente mais habilidades matemáticas.

\subsubsection{Competições online}

Existem também muitos concursos online abertos a todos.
No momento, o site de concursos mais ativo é o Codeforces,
que organiza concursos semanais.
No Codeforces, os participantes são divididos em duas divisões:
iniciantes competem em Div2 e programadores mais experientes em Div1.
Outros sites de concursos incluem AtCoder, CS Academy, HackerRank e Topcoder.

Algumas empresas organizam concursos online com finais presenciais.
Exemplos de tais concursos são Facebook Hacker Cup,
Google Code Jam e Yandex.Algorithm.
Claro, as empresas também usam esses concursos para recrutamento:
ter um bom desempenho em uma competição é uma boa maneira de provar suas habilidades.

\subsubsection{Books}

Já existem alguns livros (além deste) que focam em programação competitiva
e resolução algorítmica de problemas:

\begin{itemize}
\item S. S. Skiena and M. A. Revilla:
\emph{Programming Challenges: The Programming Contest Training Manual} \cite{ski03}
\item S. Halim and F. Halim:
\emph{Competitive Programming 3: The New Lower Bound of Programming Contests} \cite{hal13}
\item K. Diks et al.: \emph{Looking for a Challenge? The Ultimate Problem Set from
the University of Warsaw Programming Competitions} \cite{dik12}
\end{itemize}

Os primeiros dois livros são voltados para iniciantes, enquanto que o último
livro contém material avançado.

Claro, livros de algoritmos gerais também são adequados para
programadores competitivos.
Alguns livros populares são:

\begin{itemize}
\item T. H. Cormen, C. E. Leiserson, R. L. Rivest and C. Stein:
\emph{Introduction to Algorithms} \cite{cor09}
\item J. Kleinberg and É. Tardos:
\emph{Algorithm Design} \cite{kle05}
\item S. S. Skiena:
\emph{The Algorithm Design Manual} \cite{ski08}
\end{itemize}
